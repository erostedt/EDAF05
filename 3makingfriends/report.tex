\documentclass{article}
\usepackage{amsmath}
\usepackage[utf8]{inputenc}
\usepackage{booktabs}
\usepackage{microtype}
\usepackage[colorinlistoftodos]{todonotes}
\pagestyle{empty}

\title{Making friends report}
\author{Eric Rostedt \\ Mattias Pettersson}

\begin{document}
\maketitle

\section{Results}

\todo[inline]{Briefly comment the results, did the script say all your solutions were correct? Approximately how long time does it take for the program to run on the largest input? What takes the majority of the time?}

When we ran the file check\_solution.sh on our script we got that all of our solutions were correct. When running using pypy we get a total running time of about 30 seconds, where 12 seconds are in the parsing, 8 seconds to construct set of edges, 8 seconds in sorting the edges, 2 seconds for the rest of the algorithm, So the algorithm is very fast, but the slow parts are the other things. 

\section{Implementation details}

\todo[inline]{How did you implement the solution? Which data structures were used? Which modifications to these data structures were used? What is the overall running time? Why?}
We read the data from the input files using sys.stdin.readlines() and created a tree using our own-written node class. We then implemented the pseudo-code from the lecture on Kruskal's algorithm in order to create the minimum spanning tree. \\

We created a node-class that we called for every new node. Each node object contained its "name", an empty list of its potential neighbours and the weight of the "roads" to these neighbours, and an empty node-object later pointing to its predecessor.\\

We then looped over all of the lines in the file to create the tree. The next step was to start the Kruskal algorithm on this tree. The Kruskal algorithm sorts all the weights in the tree and "walks" across the lowest one, saving it. It then continues doing this, but before chosing a weight it checks that no cycles are created. It does this until all nodes are connected to eachother in some way.\\

The total running time was 36 seconds, which is considerably longer than each individual problem.\\
\end{document}
